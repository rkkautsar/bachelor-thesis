%
% Template Laporan Skripsi/Thesis 
%
% @author  Andreas Febrian, Lia Sadita 
% @version 1.03
%
% Dokumen ini dibuat berdasarkan standar IEEE dalam membuat class untuk 
% LaTeX dan konfigurasi LaTeX yang digunakan Fahrurrozi Rahman ketika 
% membuat laporan skripsi. Konfigurasi yang lama telah disesuaikan dengan 
% aturan penulisan thesis yang dikeluarkan UI pada tahun 2008.
%

%
% Tipe dokumen adalah report dengan satu kolom. 
%
\documentclass[12pt, a4paper, onecolumn, oneside, final]{report}

% Load konfigurasi LaTeX untuk tipe laporan thesis
\usepackage{_internals/uithesis}
\usepackage{enumitem}
\usepackage{bm}

% Load konfigurasi khusus untuk laporan yang sedang dibuat
%-----------------------------------------------------------------------------%
% Informasi Mengenai Dokumen
%-----------------------------------------------------------------------------%
%
% Judul laporan.
\var{\judul}{A Benchmarking Tool for Computational Reproducibility}
%
% Tulis kembali judul laporan, kali ini akan diubah menjadi huruf kapital
\Var{\Judul}{A Benchmarking Tool for Computational Reproducibility}
%
% Tulis kembali judul laporan namun dengan bahasa Ingris
\var{\judulInggris}{A Benchmarking Tool for Computational Reproducibility}

%
% Tipe laporan, dapat berisi Skripsi, Tugas Akhir, Thesis, atau Disertasi
\var{\type}{Thesis}
%
% Tulis kembali tipe laporan, kali ini akan diubah menjadi huruf kapital
\Var{\Type}{Thesis}

\var{\fullType}{Bachelor Thesis}
\Var{\FullType}{Bachelor Thesis}
%
% Tulis nama penulis
\var{\penulis}{Rakha Kanz Kautsar}
%
% Tulis kembali nama penulis, kali ini akan diubah menjadi huruf kapital
\Var{\Penulis}{Rakha Kanz Kautsar}
%
% Tulis NPM penulis
\var{\npm}{1506688784}
%
% Tuliskan Fakultas dimana penulis berada
\Var{\Fakultas}{Computer Science}
\var{\fakultas}{Computer Science}
%
% Tuliskan Program Studi yang diambil penulis
\Var{\Program}{Computer Science}
\var{\program}{Computer Science}
%
% Tuliskan tahun publikasi laporan
\Var{\bulan}{Juli}
\Var{\tahun}{2019}
%
% Tuliskan gelar yang akan diperoleh dengan menyerahkan laporan ini
\var{\gelar}{Bachelor of Computer Science}
%
% Tuliskan tanggal pengesahan laporan, waktu dimana laporan diserahkan ke
% penguji/sekretariat
\var{\tanggalPengesahan}{XX Juli 2019}
%
% Tuliskan tanggal keputusan sidang dikeluarkan dan penulis dinyatakan
% lulus/tidak lulus
\var{\tanggalLulus}{XX Juli 2019}
%
% Tuliskan pembimbing
\var{\pembimbing}{Drs. Lim Yohanes Stefanus, M.Math., Ph.D}
\var{\pembimbingDua}{Dr. Johannes K. Fichte}
%
% Alias untuk memudahkan alur penulisan paa saat menulis laporan
\var{\saya}{Penulis}
\var{\First}{We}
\var{\first}{we}
\var{\firstposs}{our}
\var{\Firstposs}{Our}

%-----------------------------------------------------------------------------%
% Judul Setiap Bab
%-----------------------------------------------------------------------------%
%
% Berikut ada judul-judul setiap bab.
% Silahkan diubah sesuai dengan kebutuhan.
%
\Var{\kataPengantar}{Preface}
\Var{\chIntroduction}{Introduction}
\Var{\chExperimentation}{Experimentation \& Benchmarking}
\Var{\chResource}{Measuring \& Limiting Resource}
\Var{\chExisting}{Existing Benchmarking Tools}
\Var{\chImplementation}{Implementation}
\Var{\chEvaluation}{Evaluation}
\Var{\chConclusion}{Conclusion \& Future Works}



\tikzset{
    rectangle with rounded corners north west/.initial=4pt,
    rectangle with rounded corners south west/.initial=4pt,
    rectangle with rounded corners north east/.initial=4pt,
    rectangle with rounded corners south east/.initial=4pt,
}
\makeatletter
\pgfdeclareshape{rectangle with rounded corners}{
    \inheritsavedanchors[from=rectangle] % this is nearly a rectangle
    \inheritanchorborder[from=rectangle]
    \inheritanchor[from=rectangle]{center}
    \inheritanchor[from=rectangle]{north}
    \inheritanchor[from=rectangle]{south}
    \inheritanchor[from=rectangle]{west}
    \inheritanchor[from=rectangle]{east}
    \inheritanchor[from=rectangle]{north east}
    \inheritanchor[from=rectangle]{south east}
    \inheritanchor[from=rectangle]{north west}
    \inheritanchor[from=rectangle]{south west}
    \backgroundpath{% this is new
        % store lower right in xa/ya and upper right in xb/yb
        \southwest \pgf@xa=\pgf@x \pgf@ya=\pgf@y
        \northeast \pgf@xb=\pgf@x \pgf@yb=\pgf@y
        % construct main path
        \pgfkeysgetvalue{/tikz/rectangle with rounded corners north west}{\pgf@rectc}
        \pgfsetcornersarced{\pgfpoint{\pgf@rectc}{\pgf@rectc}}
        \pgfpathmoveto{\pgfpoint{\pgf@xa}{\pgf@ya}}
        \pgfpathlineto{\pgfpoint{\pgf@xa}{\pgf@yb}}
        \pgfkeysgetvalue{/tikz/rectangle with rounded corners north east}{\pgf@rectc}
        \pgfsetcornersarced{\pgfpoint{\pgf@rectc}{\pgf@rectc}}
        \pgfpathlineto{\pgfpoint{\pgf@xb}{\pgf@yb}}
        \pgfkeysgetvalue{/tikz/rectangle with rounded corners south east}{\pgf@rectc}
        \pgfsetcornersarced{\pgfpoint{\pgf@rectc}{\pgf@rectc}}
        \pgfpathlineto{\pgfpoint{\pgf@xb}{\pgf@ya}}
        \pgfkeysgetvalue{/tikz/rectangle with rounded corners south west}{\pgf@rectc}
        \pgfsetcornersarced{\pgfpoint{\pgf@rectc}{\pgf@rectc}}
        \pgfpathclose
    }
}
% Daftar istilah yang mungkin perlu ditandai 
\input{istilah}

% Awal bagian penulisan laporan
\begin{document}
%
% Sampul Laporan
%
% Sampul Laporan

%
% @author  unknown
% @version 1.01
% @edit by Andreas Febrian
%

\begin{titlepage}
    \begin{center}    
        \begin{figure}
            \begin{center}
                \includegraphics[width=2.5cm]{_internals/makara.eps}
            \end{center}
        \end{figure}    
        \vspace*{0cm}
        \bo{
        	UNIVERSITAS INDONESIA\\
        }
        
        \vspace*{1.0cm}
        % judul thesis harus dalam 14pt Times New Roman
        \bo{\Judul} \\[1.0cm]

        \vspace*{2.5 cm}    
        % harus dalam 14pt Times New Roman
        \bo{\FullType}

        \vspace*{3 cm}       
        % penulis dan npm
        \bo{\Penulis} \\
        \bo{\npm} \\

        \vspace*{5.0cm}

        % informasi mengenai fakultas dan program studi
        \bo{
        	FACULTY OF \Fakultas\\
        	STUDY PROGRAM OF \Program \\
        	DEPOK \\
        	\tahun
        }
    \end{center}
\end{titlepage}


%
% Gunakan penomeran romawi
\pagenumbering{roman}

%
% load halaman judul dalam
\addChapter{TITLE PAGE}
%
% Halaman Judul Laporan 
%
% @author  unknown
% @version 1.01
% @edit by Andreas Febrian
%

\begin{titlepage}
    \begin{center}\begin{figure}
            \begin{center}
                \includegraphics[width=2.5cm]{_internals/makara.eps}
            \end{center}
        \end{figure}    
        \vspace*{0cm}
        \bo{
        	UNIVERSITAS INDONESIA\\
        }
        
        \vspace*{1.0cm}
        % judul thesis harus dalam 14pt Times New Roman
        \bo{\Judul} \\[1.0cm]

        \vspace*{2.5 cm}    
        % harus dalam 14pt Times New Roman
        \bo{\FullType} \\
        % keterangan prasyarat
        \bo{A thesis submitted in partial fulfillment of the requirements for the degree of \\
        \gelar}\\

        \vspace*{3 cm}       
        % penulis dan npm
        \bo{\Penulis} \\
        \bo{\npm} \\

        \vspace*{5.0cm}

        % informasi mengenai fakultas dan program studi
        \bo{
        	FACULTY OF \Fakultas\\
        	STUDY PROGRAM OF \Program \\
        	DEPOK \\
        	\bulan\ \tahun
        }
    \end{center}
\end{titlepage}

%
% setelah bagian ini, halaman dihitung sebagai halaman ke 2
\setcounter{page}{2}

%
% load halaman pengesahan
\addChapter{APPROVAL PAGE}
%
% Halaman Pengesahan
%
% @author  Andreas Febrian
% @version 1.01
%

\chapter*{Approval Page}

\vspace*{0.2cm}
\noindent 

\noindent
\begin{tabular}{l l p{11cm}}
	\bo{Title}&: & \judul \\ 
	\bo{Name}&: & \penulis \\
	\bo{NPM}&: & \npm \\
\end{tabular} \\

\vspace*{1.2cm}

\noindent This thesis has been examined and approved.\\[0.3cm]
\begin{center}
\tanggalPengesahan \\[2cm]


\underline{\pembimbing}\\[0.1cm]
Supervisor

\vspace{2cm}

\underline{\pembimbingDua}\\[0.1cm]
Co-supervisor
\end{center}

\newpage
%
% load halaman orisinalitas 
\addChapter{STATEMENT OF ORIGINALITY}
%
% Halaman Orisinalitas
%
% @author  Andreas Febrian
% @version 1.01
%

\chapter*{Statement of Originality}
\vspace*{2cm}

\begin{center}
	\bo{I hereby declare that this thesis is my own work,\\
	and all sources either cited or referred have been \\
	properly acknowledged.} \\
	\vspace*{2.6cm}
	
	\begin{tabular}{l c l}
	\bo{Name} & : & \bo{\penulis} \\
	\bo{NPM} & : & \bo{\npm} \\ 
	\bo{Signature} & : & \\
	& & \\
	& & \\
	\bo{Date} & : & \bo{\tanggalPengesahan} \\	
	\end{tabular}
\end{center}

\newpage
%
%
\addChapter{CERTIFICATION OF APPROVAL}
%
% Halaman Pengesahan Sidang
%
% @author  Andreas Febrian, Andre Tampubolon 
% @version 1.02
%

\chapter*{Certification of Approval}

\vspace*{0.4cm}
\noindent 

\noindent
\begin{tabular}{ll p{9cm}}
	This thesis, submitted by&: & \\
	Name&: & \penulis \\
	NPM&: & \npm \\
	Study Program&: & \program \\
	\type~Title&: & \judul \\
\end{tabular} \\

\vspace*{1.0cm}

\noindent \bo{Has successfully been defended in the presence of Board of
Examiners and accepted in partial fulfillment of the requirements for
the degree of \gelar~ at the Study Program of \program, Faculty of \fakultas,
Universitas Indonesia}\\[0.2cm]

\begin{center}
	\bo{Board of Examiners}
\end{center}

\vspace*{0.3cm}

\begin{tabular}{l l l l }
	& & & \\
	Supervisor&: & \pembimbing & (\hspace*{3.0cm}) \\
	& & & \\
	Co-supervisor&: & \pembimbingDua & (\hspace*{3.0cm}) \\
	& & & \\
	Examiner&: & & (\hspace*{3.0cm}) \\
	& & & \\
	Examiner&: & & (\hspace*{3.0cm}) \\
\end{tabular}\\

\vspace*{2.0cm}

\begin{tabular}{ll l}
	Signed at&: & Depok\\
	Date&: & \tanggalLulus \\
\end{tabular}


\newpage
%
%
\addChapter{PREFACE}
%-----------------------------------------------------------------------------%
\chapter*{Preface}
%-----------------------------------------------------------------------------%

\todo{preface}

First and foremost, we would like to thank 

\vspace*{0.1cm}
\begin{flushright}
Dresden, 14 June 2019\\[0.1cm]
\vspace*{1cm}
\penulis

\end{flushright}
%
%
\clearpage
\addChapter{STATEMENT OF CONSENT OF ACADEMIC PUBLICATION}
% 
% @author  Andre Tampubolon, Andreas Febrian
% @version 1.01
% 

\chapter*{Statement of Consent of Thesis Publication for Academic Interests}

\vspace*{0.2cm}
\noindent 
As a member of academic community of Universitas Indonesia,
I the undersigned below:
\vspace*{0.4cm}


\begin{tabular}{p{4.2cm} l p{6cm}}
	\bo{Name} & : & \penulis \\ 	
	\bo{NPM} & : & \npm \\
	\bo{Study Program} & : & \program\\	
	\bo{Faculty} & : & \fakultas\\
	\bo{Academic Work} & : & \type~(S1) \\
\end{tabular}

\vspace*{0.6cm}
\noindent in the scientific development interests, hereby consent to grant
Universitas Indonesia a \bo{\textit{Non‐exclusive Royalty‐Free Right}}
my scientific work entitled:

\begin{center}
	\judul
\end{center}

\noindent , along with its existing sets of supplementary materials (if required).
With Non‐exclusive Royalty‐Free Right, Universitas Indonesia has rights
to save, media‐transfer/format, manage in a database, maintain, and
publish my final project provided my name is retained as the author/creator
and the owner of Copyright. \\

\noindent I certify that the statement is true to the best of my knowledge

\begin{center}
	\vspace*{0.8cm}
	\begin{tabular}{rl}
		Defined at : & Depok \\
		Date : & \tanggalPengesahan \\
	\end{tabular}\\

	\vspace*{0.2cm}
	Signed \\
	\vspace*{1.1cm}
	(\penulis)
\end{center}

\newpage


%
% 
\addChapter{ABSTRACT}
%
% Halaman Abstract
%
% @author  Andreas Febrian
% @version 1.00
%

\chapter*{Abstract}

\vspace*{0.2cm}
{
	\setlength{\parindent}{0pt}

	\begin{tabular}{@{}l l p{10cm}}
		Name&: & \penulis \\
		Study Program&: & \program \\
		Title&: & \judulInggris \\
		Supervisors&: & \pembimbing \\
				   &  & \pembimbingDua \\
	\end{tabular}

	\bigskip
	\bigskip

	Benchmarking is one popular way to conduct experiments, especially in the field of computational science.
	It provides a systematic, easily reproducible framework to compare performance between implementations.
	Benchmarking tool uses resource monitoring tool to measure the performance of selected implementations to selected test instances.
	We review the existing benchmarking tools and found none of them meet our requirements of an ideal benchmarking tool.
	We thus propose a new benchmarking tool designed for reproducible research.
	We also provide a cross-platform implementation that has been tested to run in Linux and macOS.
	Our evaluation shows that the proposed benchmarking tool fulfills most of the requirements, except for comprehensive documentation and testing.
	The proposed tool also shows potential in future development to further lower the effort to embrace reproducibility.

	\bigskip

	Key words:\\
	Benchmarking, distributed computing, computational science, reproducibility
}

\newpage
%
%
%
% Halaman Abstrak
%
% @author  Andreas Febrian
% @version 1.00
%

\chapter*{Abstrak}

\vspace*{0.2cm}
{
	\setlength{\parindent}{0pt}
	
	\begin{tabular}{@{}l l p{10cm}}
		Nama&: & \penulis \\
		Program Studi&: & \program \\
		Judul&: & \judulIndo \\
		Pembimbing&: & \pembimbing \\
				  &  & \pembimbingDua \\
	\end{tabular}

	\bigskip
	\bigskip

	\emph{Benchmarking} adalah salah satu cara yang populer untuk melakukan eksperimen, terutama di bidang \emph{computational science}.
	\emph{Benchmarking} menyediakan kerangka kerja yang sistematis dan \emph{reproducible} untuk membandingkan performa antara implementasi.
	Alat \emph{benchmarking} menggunakan alat \emph{resource monitoring} untuk mengukur performa dari implementasi tertentu pada data tertentu.
	Di skripsi ini, implementasi alat \emph{resource monitoring} yang \emph{cross-platform} dan telah diuji di Linux dan macOS disediakan setelah melihat bahwa tidak ada alat yang berjalan di macOS.
	Alat-alat \emph{benchmarking} yang sudah ada ditinjau kemudian ditemukan bahwa tidak ada yang memenuhi semua kriteria yang ditentukan.
	Oleh karena itu, alat \emph{benchmarking} baru yang didesain untuk mendukung \emph{reproducibility} diajukan.
	Implementasi \emph{cross-platform} dari alat ini juga disediakan dan sudah diuji berjalan di Linux dan macOS.
	Evaluasi menunjukkan bahwa alat \emph{benchmarking} yang diajukan memenuhi hampir semua kriteria yang ditentukan kecuali pada poin adanya dokumentasi yang komprehensif dan pengujian.
	Alat yang diajukan menunjukkan potensi yang besar pada pengembangan selanjutnya untuk lebih menurunkan upaya untuk mencapai \emph{reproducibility}.

	\bigskip

	Kata kunci:\\
	\emph{Benchmarking}, \emph{resource monitoring}, komputasi terdistribusi, \emph{computational science}, \emph{reproducibility}
}

\newpage

%
% Daftar isi, gambar, dan tabel
%
\phantomsection
\tableofcontents
\clearpage
\phantomsection
\listoffigures
\clearpage
\phantomsection
\listoftables
\clearpage
\phantomsection
\addcontentsline{toc}{chapter}{\uppercase{List of Listings}}
\listoflistings
\clearpage

%
% Gunakan penomeran Arab (1, 2, 3, ...) setelah bagian ini.
%
\pagenumbering{arabic}

%
%
%
%-----------------------------------------------------------------------------%
\chapter{\babSatu}
%-----------------------------------------------------------------------------%


%-----------------------------------------------------------------------------%
\section{Overview}
%-----------------------------------------------------------------------------%

In computational science, it is often preferable to compare new algorithm to previous studies and get various measurements. But unfortunately most of the time it is not easy to reproduce those studies. \cite{collbergRepeatabilityComputerSystems2016} has shown that out of 402 computer science and engineering paper backed by code that they have examined, only 32.3\% can be built in an under 30 minutes attempt to resolve dependencies and environments needed to run the code, and this number only raises to 48.3\% when the attempt time is not limited. Not to mention that it is only 56,22\% out of those 402 paper whose source code is obtainable, even after requesting directly from the authors.

Few attempts has been made to this computational reproducibility problem. Some notable examples are Sacred Infrastructure \citep{greffSacredInfrastructureComputational2017} on reproducible experiment, Reprozip \citep{chirigatiReproZipComputationalReproducibility2016} attempt on packing provenance, and BenchExec \citep{beyerReliableBenchmarkingRequirements2019} attempt on reliable benchmarks. Out of those, only BenchExec tackle the problem of benchmarking, but even so it is too domain-specific on software verification and doesn't support running long-running benchmarks such as those with millions of algorithms/parameters/instance combinations in high performance computing (HPC) clusters.

A new benchmarking tool that is capable of  limiting and measuring resource usage, parallel runs, running on HPC clusters, (partially) re-run the benchmarks with new algorithm version, and producing reproducible result that can be shared with others will surely be a huge contribution to computational science in general. Authors can benchmark their algorithms with various parameters and compare them to previous algorithms. Reviewers can then check the benchmark results claimed, and other researchers can compare their algorithms by extending from this benchmark results in an objective manner.

%-----------------------------------------------------------------------------%
\section{Contributions}
%-----------------------------------------------------------------------------%

%-----------------------------------------------------------------------------%
\section{Research Scope}
%-----------------------------------------------------------------------------%


%-----------------------------------------------------------------------------%
\section{Outline}
%-----------------------------------------------------------------------------%
Sistematika penulisan laporan adalah sebagai berikut:
\begin{itemize}
	\item Bab 1 \babSatu \\
	\item Bab 2 \babDua \\
	\item Bab 3 \babTiga \\
	\item Bab 4 \babEmpat \\
	\item Bab 5 \babLima \\
	\item Bab 6 \babEnam \\
	\item Bab 7 \kesimpulan \\
\end{itemize}

\chapter{\babDua}

This chapter [...]. \hyperref[sec:experimentation]{Section 2.1} discuss [...]

\section{Experimentation}
\label{sec:experimentation}

Science aims to understand why things happen as they do in the natural world \citep{careyBeginnerGuideScientific2012}
Investigating these why questions can be achieved by following what is known as the scientific method, or what Karl Popper has defined as the hypothetico-deductive method, a combination of inductive and deductive reasoning \citep{wallimanResearchMethodsBasics2010a}.
He further defines the scientific method as follows:
\begin{itemize}[noitemsep]
	\item identification or clarification of a problem;
	\item developing a hypothesis (testable theory) inductively from observations;
	\item charting their implications by deduction;
	\item practical or theoretical testing of the hypothesis;
	\item rejecting or refining it in the light of the results.
\end{itemize}

\citet{careyBeginnerGuideScientific2012} provides a simpler definition for the scientific method as three simple steps: observing, explaining, and testing.
These definitions suggest that science advances by the means of trial and error.
The explaining-testing steps are repeated until the result is acceptable.
When a theory is falsified, another one is proposed and tested, until the most fitting theory is accepted \citep{wallimanResearchMethodsBasics2010a}.

Experimentation is one way to test these proposed theories or hypotheses.
Some would argue that as experiments don't proof anything.
But on the other hand, experiments can falsify a theory and corroborate it, but not actually proofing that the theory is true \citep{tichyShouldComputerScientists1998}.
Thus, experiments can be seen as just another way of proofing theories by \textit{reductio ad absurdum}, like what proof by contradiction is commonly used in logic and mathematics.

Experimentation can also be used for induction, deriving generalized conclusion from repeated observations.
A rather well-known example in computer science is the case of artificial neural networks.
Experiments demonstrated that its performance is better than what was predicted after being discarded in theoretical grounds \citep{tichyShouldComputerScientists1998}.

Another example is Boolean Satisfiability (SAT).
It has been proved to be $\mathcal{NP}$-complete by the Cook-Levin theorem.
Many would give up after seeing that the problem they are trying to solve is in $\mathcal{NP}$.
But this is not the case.
State-of-the-art SAT solvers are good enough in solving real world problems with millions of literals and clauses.
It is widely used in many fields despite the fact that the problem itself is seen as bad in theory.

Not only in computer science, experimentation has been used in practice in many fields.
In particular fields like physics and chemistry that involves observing and explaining physical objects often use this method.

An early example would be the famous thought experiment by Galileo Galilei.
The thought experiment was about tying two bodies of object, one lighter and one heavier and dropping if from top of a tower.
If, like many believed in that time, heavier object will fall faster as such the case with a heavy metal ball and a feather, the lighter one will pull the heavier one and slow down the fall.
But these tied objects makes an object of heavier than either and thus should fall faster, this makes it a contradiction \citep{goodmanWhatDoesResearch2016}. This example also points out that external factors which are not taken into account can effect the result of an experiment in a bad way, such as air resistance in this case.

There's also a rather extreme case of measuring the weight of a human soul, conducted by Duncan MacDougall, MD. He suggest that human soul has a measurable mass. To proof this hypothesis, he use six terminal patients and weighed them before, during, and after the process of death \citep{ryanModernExperimentalDesign2007a}.

And last but not least, there's the expensive validation by Isaac Eddington in 1919 to prove Einstein's theory that gravity bends light.
This important experiment involves expedition to Principe Island, West Africa and pushed the limits of photographic emulsion technology \citep{tichyShouldComputerScientists1998}.
And now, 100 years after the validation, it was further supported by another great advancement in astronomy.
In April 10th 2019, the Event Horizon Telescope (EHT) presented to the public the first ever successful image of a black hole, an object predicted by Einstein's general theory of relativity.

Besides for the objectives of validation and induction already mentioned above, there's a lot of reason to run an experiment.
\citet{montgomeryDesignAnalysisExperiments2013} mentioned some of the reasons as follows:
\begin {itemize}
	\item \textbf{Factor screening or characterization}. This is often the case when working with new systems or technologies to minimize wasted resource usage.
	\item \textbf{Optimization}. Finding the desirable settings of factors that result in desirable response.
	\item \textbf{Confirmation}. This is more or less the same as the validation objective already mentioned.
	\item \textbf{Discovery}. This is more or less the same as the induction objective already mentioned.
	\item \textbf{Robustness}. Finding the condition on which the response variables will degrade.
\end{itemize}

\citet{montgomeryDesignAnalysisExperiments2013} also defines a guideline in designing an experiment that maps nicely to the scientific method:

\begin{enumerate}[noitemsep]
	\item Recognition of and statement of the problem
	\item Selection of the response variable
	\item Choice of factors, levels, and ranges
	\item Choice of experimental design
	\item Performing the experiment
	\item Statistical analysis of the data
	\item Conclusions and recommendations
\end{enumerate}

Step 1-3 can be considered as the pre-experimental planning steps.
He also noted that steps 2 and 3 is interchangeable in practice.
Furthermore, the experiment should also be done in an iterative way.

\section{Reproducible Benchmarks}

The term reproducibility is often intertwined with the term replicability.
\citet{drummondReplicabilityNotReproducibility2009} argues that replicability is the impoverished version of reproducibility which albeit the great effort to achieve it, its purpose is only for preventing fraud.

\citet{goodmanWhatDoesResearch2016} suggests to define new terminologies for reproducibility to avoid the inconsistencies between replicability and reproducibility:
\begin{itemize}
	\item \textbf{Methods reproducibility}. Ability to implement, as exactly as possible, the experimental and computational procedures, with the same data and tools, to obtain the same results.

	\item \textbf{Results reproducibility}. The production of corroborating results in a new study, having followed the same experimental methods.

	\item \textbf{Inferential reproducibility}. The making of knowledge claims of similar strength from a study replication or reanalysis.
\end{itemize}

\citet{gundersenStateArtReproducibility2018} remarks that methods reproducibility is what Drummond states by replicability, and the combination of both results and inferential reproducibility is what Drummond states by reproducibility.
They also suggest three degrees of reproducibility, specific to the Artificial Intelligence (AI) field.
While these are defined as specific in the AI field, it can be applied to computational science in general with a slight modification:
\begin{itemize}
	\item \textbf{R1: Experiment Reproducible}. The results of an experiment are experiment reproducible when the execution of the same implementation produces the same results when executed on the same data.
	\item \textbf{R2: Data Reproducible}. The results of an experiment are data reproducible when an experiment is conducted that executes an alternative implementation that produces the same results when executed on the same data.
	\item \textbf{R3: Method Reproducible}. The results of an experiment are method reproducible when the execution of an alternative implementation produces the same results when executed on different data.
\end{itemize}

In short, R1 is the highest degree of reproducibility.
R3 only requires documented method of the experiment, while R2 additionally requires documented data used in the experiment, and finally R1 also requires documented experiment implementation.

They also reported that no papers are fully reproducible by any degree defined above.
The degree of reproducibility of each paper (measured by the number of required variables documented) is only in the 24-26\% range.
\citet{collbergRepeatabilityComputerSystems2016} results further support this issue.
The degree of reproduciblity in computer science is only up to 54\% even after considerable effort in contacting the authors.
They mention some of the reasons as licensing issues, no version tracking, the code is not ready for public, no backup, obsolete dependencies, and so on.

\citet{beyerReliableBenchmarkingRequirements2019} defines benchmarking as a performance evaluation method that is used for comparing different tools of the same domain, evaluating and comparing different features or configurations of a tool, or finding out the performance of a tool under different inputs.
This suggests that benchmarking is a good option for experiments where the design chosen is comparative or variance analysis.

Simplified process of benchmarking maps closely to designing an experiment, except in computational science the perform and collection phase can (and should) be automated:
\begin{enumerate}[noitemsep]
	\item Planning
	\item Performing benchmarks \& collecting results
	\item Analysis
	\item Conclusion (i.e. presentation)
\end{enumerate}


\chapter{\babTiga}
\label{ch:priorWorks}

% intro

\section{Overview}

There are six benchmarking tool that \first~evaluate.
This is by no means an exhaustive collection but it should represent the current state of existing benchmarking tools.
Two of them---\textsc{Optil.io} and \textsc{StarExec}---took a web-based approach, where the user submit the configuration and the run happened in the host system.
\textsc{Optil.io} is a bit different since it specializes in competition-style benchmarking.
The rest of them---\textsc{benchmark-tool}, \textsc{BenchExec}, \textsc{BenchKit}, and \textsc{JuBE}---approaches benchmarking as task to be run in the local machine or submitted to a cluster system.


\section{Evaluation method}

The requirements from \hyperref[sec:idealBenchmarkingTool]{Section 2.3} are broken down to a few key factors.
These key factors, stated as questions, are used to measure how much of the requirements are fulfilled.

\newcounter{reqCount}
\newcounter{reqFactorCount}[reqCount]
\newcommand{\reqLabel}[1]{
	\setcounter{reqFactorCount}{0}
	\addtocounter{reqCount}{1}
	\arabic{reqCount}.
	#1
}
\newcommand{\reqFactor}[1]{
	\addtocounter{reqFactorCount}{1}
	(\alph{reqFactorCount}) #1
}

\begin{ThreePartTable}
	\begin{TableNotes}
		\footnotesize
		\item[$\alpha$] Subjective evaluation
	\end{TableNotes}
	\begin{longtable}{ll}

		\textbf{Requirements} & \textbf{Factors} \\
		\toprule
		\endhead

		% \cmidrule{2-2}
		\multicolumn{2}{r}{\textit{continued}}
		\endfoot

		\bottomrule
		\insertTableNotes\\
		\caption{Metrics for evaluating existing benchmarking tools}
		\endlastfoot

		\multirow{5}{*}{\reqLabel{Extensibility}}
			& \reqFactor{Flexible evaluation step} \\*
			& \reqFactor{Flexible analysis step} \\*
			& \reqFactor{Flexible benchmark instance source} \\*
			& \reqFactor{Does not enforce implementation type} \\*
			& \reqFactor{Can support arbitrary task scheduler} \\*
		\midrule

		\multirow{5}{*}{\reqLabel{Configurability}}
			& \reqFactor{Multiple runs} \\*
			& \reqFactor{Multiple tool configurations} \\*
			& \reqFactor{Support parameter space} \\*
			& \reqFactor{Benchmark instance selection} \\*
			& \reqFactor{Set resource limit} \\*
		\midrule

		\multirow{5}{*}{\reqLabel{Documentation}}
			& \reqFactor{Self-documenting configuration} \\*
			& \reqFactor{Installation guide} \\*
			& \reqFactor{Configuration guide} \\*
			& \reqFactor{Main workflow guide} \\*
			& \reqFactor{Comprehensive documentation\tnote{$\alpha$}} \\*
		\midrule

		\multirow{4}{*}{\reqLabel{Setup Effort}}
			& \reqFactor{No superuser privilege} \\*
			& \reqFactor{Installation guide} \\*
			& \reqFactor{Documented requirements} \\*
			& \reqFactor{No cumbersome dependencies\tnote{$\alpha$}} \\*
		\midrule

		\multirow{6}{*}{\reqLabel{Accuracy \& Reliability}}
			& \reqFactor{Measure and Limit Resources Accurately} \\*
			& \reqFactor{Terminate Processes Reliably} \\*
			& \reqFactor{Assign Cores Deliberately} \\*
			& \reqFactor{Respect Nonuniform Memory Access} \\*
			& \reqFactor{Avoid Swapping} \\*
			& \reqFactor{Isolate Individual Runs} \\*
		\midrule

		\multirow{5}{*}{\reqLabel{Reproducibility}}
			& \reqFactor{Stored system information} \\*
			& \reqFactor{Sharable results} \\*
			& \reqFactor{Sharable configuration} \\*
			& \reqFactor{Encourage sharable data\tnote{$\alpha$}} \\*
			& \reqFactor{Encourage sharable implementation\tnote{$\alpha$}} \\*
		\bottomrule
	\end{longtable}
\end{ThreePartTable}


\section{Evaluation}

Each of the evaluated tools are scored according to the degree of fulfillment for each requirements.
That is, if $M_i$ is the scored degree of fulfillment of the $i$-th requirement, and $\mu_{i}$ is a vector of size $n$ such that
\[
	\mu_{ij} =
	\begin{cases}
		1 & \text{if the $j$-th key factor of $i$-th requirement is fulfilled}\\
		0 & \text{otherwise}
	\end{cases}
\]
then
\[
	M_i = \frac{\sum\mu_{i}}{|\mu_i|}
\]

Furthermore, as an overall measure, \first~also denote $M$ as the weighted sum of all $M_i$:
\[
	M = \sum{\frac{M_i |\mu_i|}{|\mu_i|}}
\]

\subsection{\textsc{benchmark-tool}}
\subsection{\textsc{BenchExec}}
\subsection{\textsc{Benchkit}}
\subsection{\textsc{JuBE}}
\subsection{\textsc{Optil.io}}
\subsection{\textsc{StarExec}}

\chapter{\babEmpat}

\section{Overview}

To objectively compare two or more programs, an objective measurement is needed.
Resource consumption like CPU time elapsed, CPU instruction used, or peak memory usage is often considered as the go-to measurement for benchmarking in computational science.
The resources measured might include information regarding processes, threads, computation, memory, input/output, and files of a program \citep{juvePracticalResourceMonitoring2015}.

The measurement of these resource consumption is not only used in benchmarking.
Some of its usage includes but not limited to:
its daily usage in user program such as \textit{Task Manager} in Windows or \code{top} in Linux,
judging whether a program passes some threshold marks in education or competition field,
measuring the efficiency of a job scheduling system in High Throughput Computing (HTC) field [need citation],
selecting dataset for a competition,
and getting best-enough result from an iterative optimization algorithm after some desired time.
Because of this wide area of usage, there are many attempts to implement these measurement to achieve the best result.

\section{Monitoring Mechanism}

\citet{juvePracticalResourceMonitoring2015} distinguish resource monitoring into three different mechanism.
There are tradeoffs between these mechanisms and so it's often preferred to use a combination of more than one mechanism to achieve better results.

\subsection{Query}

The query approach works by querying resource usage information directly from the operating system.
To monitor the resource usage over time, this means the query is executed at some interval, effectively doing a polling mechanism.
More frequent polling will result in a more timely information but the overhead also increases.
Querying is often the easiest resource monitoring mechanism in term of implementation \citep{juvePracticalResourceMonitoring2015}.

It is the least intrusive mechanism, albeit the information received from this mechanism also immediately expires.
A resource usage surge could happen in between the queries, so this mechanism is not accurate.
In short, the query mechanism provides an easy to implement but inaccurate measurement of the resource monitored.

\code{getrusage()}\footnote{see \href{https://linux.die.net/man/2/getrusage}{\code{man getrusage}}} is a POSIX standard system call.
Unfortunately, this standard only specifies \code{ru\_utime} and \code{ru\_stime}, the user mode time spent and kernel mode time spent, respectively.
In practice, \code{getrusage()} includes more information, for example the \code{ru\_maxrss}, indicating the maximum resident set size.

A Linux specific feature that is often used for querying resource usage is the procfs\footnote{see \href{https://linux.die.net/man/5/proc}{\code{man 5 proc}}} pseudo-file system.
The information provided includes user time, system time, resident set size, number of threads, and many more.
On a side note, since the operating system always account these informations for all running processes, the clock resolution used is not too accurate.
The user time and system time in \code{/proc/[pid]/stat} is measured in system ticks, which turns out to be 10 milliseconds in a typical Linux system.

A more universal query mechanism is using performance counters.
Most system nowadays---disregarding its operating system---has a hardware clock that has an incredibly accurate clock resolution, some even achieving nanoseconds resolution.
Querying this clock at the start and end of a program could measure what is called the wall clock time.
This measurement is less informative than CPU times measured by user time and system time, but still useful in spite of that.

The library \code{psutil}\footnote{\href{https://github.com/giampaolo/psutil}{https://github.com/giampaolo/psutil}} provides an abstraction for various resource queries that works in many operating systems.
This allows a cross-platform resource monitoring tool based on query mechanism to be developed.
As of now however, \first~can't find the relevant resource monitoring tool making use of this abstraction.

It is also need to be noted that this query mechanism cannot work with process tree reliably.
Because of the nature of polling, short living processes can be missed and not accounted to the final result.
This is a strong restriction because most of the computation in HTC or computational science in general often make use of parallelism or concurrency.


\subsection{Notification}

A more reliable mechanism to query a resource usage is to ask the system itself to report the usage on specific events.
This also produces less overhead compared to the query mechanism, although the information queried also immediately expires \citep{juvePracticalResourceMonitoring2015}.

\code{wait4()}\footnote{see \href{https://linux.die.net/man/2/wait4}{\code{man wait4}}} is a system call available in most UNIX that waits for a child process (and blocks the process calling this system call), then returns its \code{getrusage()} information when the process exits.
This one example of the notification mechanism in practice.

Another example is the \code{inotify}\footnote{see \href{https://linux.die.net/man/7/inotify}{\code{man inotify}}} application programming interface (API) in Linux.
This API allows one to listen for file system events, such as new file or directory created, existing files edited, a file is accessed, or a file is deleted.
Most operating system also have APIs similar to this, such as \code{fsevents} in OS X. Watchman\footnote{\href{https://facebook.github.io/watchman/}{https://facebook.github.io/watchman/}} is an open source tool by Facebook that abstracts these file system notification APIs.

Another useful notification is \code{forkstat}\footnote{\href{http://manpages.ubuntu.com/manpages/cosmic/en/man8/forkstat.8.html}{http://manpages.ubuntu.com/manpages/cosmic/en/man8/forkstat.8.html}}.
This tool notify system-wide \code{fork()}, \code{exec()}, and \code{exit()} system call activities.
Unfortunately, this tool needs superuser privilege because uses Linux netlink connector, a special socket for communication between kernel and user space.

A more powerful albeit intrusive notification can be achieved by using the UNIX \code{ptrace()}\footnote{see \href{https://linux.die.net/man/2/ptrace}{\code{man ptrace}}} system call.
\code{ptrace()}, often use for debugging, allows a process to intercept and modify system calls.
When a system call occurs, the kernel will check if the process is being traced, and if so, will notify the tracer.
With this system call one can observe, for example, the \code{fork()}, \code{exec()}, and \code{exit()} system call to track process trees.

Combining this notifications with the earlier query mechanism can result in a more powerful resource monitoring method.
For example, one can watch for filesystem events on a specific part of the filesystem, such as the \code{/tmp}, \code{/var}, or a specific work directory of the running application. Then when an event occurs, a procfs query is executed to catch the short-living process causing the event is executed.
This effectively allows the procfs query method to work more reliably compared to blindly polling the pseudo-file system.

\subsection{Interposition}



\section{More Reliable Measurement}

\subsection{Virtualization}

\subsection{Containerization}


\section{Implementations}

We consider some existing implementation and discuss their method of measuring and limiting resource.

\subsection{\textsc{runsolver}}
\subsection{\textsc{runexec}}
\subsection{\textsc{kickstart}}
\subsection{\textsc{timeout}}
\subsection{\textsc{nsjail}}
\subsection{\textsc{isolate}}
\subsection{\textsc{psmon}}
\input{src/01-body/05-bab5}
\input{src/01-body/99-kesimpulan}

%
% Daftar Pustaka
%\input{pustaka}

% Alternatif manajemen daftar pustaka dengan \bibliography
\bibliographystyle{apalike}
\bibliography{pustaka}

%
% Lampiran 
%
% \begin{appendix}
% 	\input{_internals/markLampiran}
% 	\setcounter{page}{2}\textbf{}
% 	%-----------------------------------------------------------------------------%
\addChapter{Appendix 1}
\chapter*{Appendix 1: Source Code}
%-----------------------------------------------------------------------------%

% \captionof*{listing}{examples/sat/sat/validate.py}
% \inputminted{python}{assets/listings/reprobench/examples/sat/sat/validate.py}
% \captionof*{listing}{examples/sat/tools/\_\_init\_\_.py}
% \inputminted{python}{assets/listings/reprobench/examples/sat/tools/__init__.py}
% \captionof*{listing}{examples/sat/tools/glucose.py}
% \inputminted{python}{assets/listings/reprobench/examples/sat/tools/glucose.py}
% \captionof*{listing}{examples/sat/tools/lingeling.py}
% \inputminted{python}{assets/listings/reprobench/examples/sat/tools/lingeling.py}
% \captionof*{listing}{examples/sudokusat/sudoku/validate.py}
% \inputminted{python}{assets/listings/reprobench/examples/sudokusat/sudoku/validate.py}
% \captionof*{listing}{examples/sudokusat/tools/\_\_init\_\_.py}
% \inputminted{python}{assets/listings/reprobench/examples/sudokusat/tools/__init__.py}
% \captionof*{listing}{examples/sudokusat/tools/sudoku\_team1.py}
% \inputminted{python}{assets/listings/reprobench/examples/sudokusat/tools/sudoku_team1.py}

% \captionof*{listing}{reprobench/\_\_init\_\_.py}
% \inputminted{python}{assets/listings/reprobench/reprobench/__init__.py}


% \captionof*{listing}{reprobench/console/decorators.py}
% \inputminted{python}{assets/listings/reprobench/reprobench/console/decorators.py}

\captionof*{listing}{reprobench/console/main.py}
\inputminted{python}{assets/listings/reprobench/reprobench/console/main.py}

\captionof*{listing}{reprobench/console/status.py}
\inputminted{python}{assets/listings/reprobench/reprobench/console/status.py}

\captionof*{listing}{reprobench/core/analyzer.py}
\inputminted{python}{assets/listings/reprobench/reprobench/core/analyzer.py}

\captionof*{listing}{reprobench/core/base.py}
\inputminted{python}{assets/listings/reprobench/reprobench/core/base.py}

% \captionof*{listing}{reprobench/core/bootstrap/\_\_init\_\_.py}
% \inputminted{python}{assets/listings/reprobench/reprobench/core/bootstrap/__init__.py}

\captionof*{listing}{reprobench/core/bootstrap/client.py}
\inputminted{python}{assets/listings/reprobench/reprobench/core/bootstrap/client.py}

\captionof*{listing}{reprobench/core/bootstrap/server.py}
\inputminted{python}{assets/listings/reprobench/reprobench/core/bootstrap/server.py}

\captionof*{listing}{reprobench/core/db.py}
\inputminted{python}{assets/listings/reprobench/reprobench/core/db.py}

% \captionof*{listing}{reprobench/core/events.py}
% \inputminted{python}{assets/listings/reprobench/reprobench/core/events.py}

% \captionof*{listing}{reprobench/core/exceptions.py}
% \inputminted{python}{assets/listings/reprobench/reprobench/core/exceptions.py}

\captionof*{listing}{reprobench/core/observers.py}
\inputminted{python}{assets/listings/reprobench/reprobench/core/observers.py}

\captionof*{listing}{reprobench/core/schema.py}
\inputminted{python}{assets/listings/reprobench/reprobench/core/schema.py}

\captionof*{listing}{reprobench/core/server.py}
\inputminted{python}{assets/listings/reprobench/reprobench/core/server.py}

\captionof*{listing}{reprobench/core/sysinfo.py}
\inputminted{python}{assets/listings/reprobench/reprobench/core/sysinfo.py}

\captionof*{listing}{reprobench/core/worker.py}
\inputminted{python}{assets/listings/reprobench/reprobench/core/worker.py}

% \captionof*{listing}{reprobench/executors/\_\_init\_\_.py}
% \inputminted{python}{assets/listings/reprobench/reprobench/executors/__init__.py}

\captionof*{listing}{reprobench/executors/base.py}
\inputminted{python}{assets/listings/reprobench/reprobench/executors/base.py}

\captionof*{listing}{reprobench/executors/db.py}
\inputminted{python}{assets/listings/reprobench/reprobench/executors/db.py}

% \captionof*{listing}{reprobench/executors/events.py}
% \inputminted{python}{assets/listings/reprobench/reprobench/executors/events.py}

\captionof*{listing}{reprobench/executors/psmon.py}
\inputminted{python}{assets/listings/reprobench/reprobench/executors/psmon.py}

% \captionof*{listing}{reprobench/managers/\_\_init\_\_.py}
% \inputminted{python}{assets/listings/reprobench/reprobench/managers/__init__.py}

\captionof*{listing}{reprobench/managers/base.py}
\inputminted{python}{assets/listings/reprobench/reprobench/managers/base.py}

\captionof*{listing}{reprobench/managers/local/\_\_init\_\_.py}
\inputminted{python}{assets/listings/reprobench/reprobench/managers/local/__init__.py}

\captionof*{listing}{reprobench/managers/local/manager.py}
\inputminted{python}{assets/listings/reprobench/reprobench/managers/local/manager.py}

\captionof*{listing}{reprobench/managers/slurm/\_\_init\_\_.py}
\inputminted{python}{assets/listings/reprobench/reprobench/managers/slurm/__init__.py}

\captionof*{listing}{reprobench/managers/slurm/manager.py}
\inputminted{python}{assets/listings/reprobench/reprobench/managers/slurm/manager.py}

% \captionof*{listing}{reprobench/managers/slurm/utils.py}
% \inputminted{python}{assets/listings/reprobench/reprobench/managers/slurm/utils.py}

% \captionof*{listing}{reprobench/statistics/plots/\_\_init\_\_.py}
% \inputminted{python}{assets/listings/reprobench/reprobench/statistics/plots/__init__.py}

\captionof*{listing}{reprobench/statistics/plots/base.py}
\inputminted{python}{assets/listings/reprobench/reprobench/statistics/plots/base.py}

\captionof*{listing}{reprobench/statistics/plots/cactus/\_\_init\_\_.py}
\inputminted{python}{assets/listings/reprobench/reprobench/statistics/plots/cactus/__init__.py}

% \captionof*{listing}{reprobench/statistics/plots/cactus/template.ipynb}
% \inputminted{python}{assets/listings/reprobench/reprobench/statistics/plots/cactus/template.ipynb}

% \captionof*{listing}{reprobench/statistics/tables/\_\_init\_\_.py}
% \inputminted{python}{assets/listings/reprobench/reprobench/statistics/tables/__init__.py}

\captionof*{listing}{reprobench/statistics/tables/base.py}
\inputminted{python}{assets/listings/reprobench/reprobench/statistics/tables/base.py}

\captionof*{listing}{reprobench/statistics/tables/run.py}
\inputminted{python}{assets/listings/reprobench/reprobench/statistics/tables/run.py}

% \captionof*{listing}{reprobench/task\_sources/base.py}
% \inputminted{python}{assets/listings/reprobench/reprobench/task_sources/base.py}

\captionof*{listing}{reprobench/task\_sources/doi/\_\_init\_\_.py}
\inputminted{python}{assets/listings/reprobench/reprobench/task_sources/doi/__init__.py}

% \captionof*{listing}{reprobench/task\_sources/doi/base.py}
% \inputminted{python}{assets/listings/reprobench/reprobench/task_sources/doi/base.py}

\captionof*{listing}{reprobench/task\_sources/doi/zenodo.py}
\inputminted{python}{assets/listings/reprobench/reprobench/task_sources/doi/zenodo.py}

\captionof*{listing}{reprobench/task\_sources/file.py}
\inputminted{python}{assets/listings/reprobench/reprobench/task_sources/file.py}

\captionof*{listing}{reprobench/task\_sources/url.py}
\inputminted{python}{assets/listings/reprobench/reprobench/task_sources/url.py}

\captionof*{listing}{reprobench/tools/executable.py}
\inputminted{python}{assets/listings/reprobench/reprobench/tools/executable.py}

% \captionof*{listing}{reprobench/utils.py}
% \inputminted{python}{assets/listings/reprobench/reprobench/utils.py}
% \end{appendix}

\end{document}