%-----------------------------------------------------------------------------%
\chapter*{Preface}
%-----------------------------------------------------------------------------%
Template ini disediakan untuk orang-orang yang berencana menggunakan 
\latex~untuk membuat dokumen tugas akhirnya. 
Mengapa \latex? 
Ada banyak hal mengapa menggunakan \latex, diantaranya:

\begin{enumerate}
	\item \latex~membuat kita jadi lebih fokus terhadap isi dokumen, bukan 
		tampilan atau halaman. 
	\item \latex~memudahkan dalam penulisan persamaan matematis. 
	\item Adanya automatis dalam penomoran caption, bab, subbab, subsubbab, 
		referensi, dan rumus. 
	\item Adanya automatisasi dalam pembuatan daftar isi, daftar gambar, dan
		daftar tabel. 
	\item Adanya kemudahan dalam memberikan referensi dalam tulisan dengan 
		menggunakan label. Cara ini dapat meminimalkan kesalahan pemberian 
		referensi. 
\end{enumerate}

Template ini bebas digunakan dan 
didistribusikan sesuai dengan aturan \license, yang secara sederhana berisi: 

\pic diambil dari 
\url{http://creativecommons.org/licenses/by-nc-sa/1.0/deed.en_CA}. 
Jika ingin mengentahui lebih lengkap mengenai \license, silahkan buka 
\url{http://creativecommons.org/licenses/by-nc-sa/1.0/legalcode}. 
Seluruh dokumen yang dibuat dengan menggunakan template ini sepenuhnya 
menjadi hak milik pembuat dokumen dan bebas didistribusikan sesuai dengan 
keperluan masing-masing. 
Lisensi hanya berlaku jika ada orang yang membuat template baru dengan 
menggunakan template ini sebagai dasarnya. 

Dokumen ini dibuat dengan \latex~juga. Untuk meyakinkan Anda, coba lihat 
properti dari dokumen ini dan Anda akan menemukan bagian seperti. 
Dokumen ini dimaksudkan untuk memberikan gambaran kepada Anda seperti apa 
mudahnya menggunakan \latex~dan juga memperlihatkan betapa bagus dokumen 
yang dihasilkan. 
Seluruh url yang Anda temukan dapat Anda klik. 
Seluruh referensi yang ada juga dapat diklik. 
Untuk mengerti template yang disediakan, Anda tetap harus membuka kode 
\latex~dan bermain-main dengannya. 
Penjelasan dalam PDF ini masih bersifat gambaran dan tidak begitu 
mendetail, dapat dianggap sebagai pengantar singkat. 
Jika Anda merasa kesulitan dengan template ini, mungkin ada baiknya 
Anda belajar sedikit dasar-dasar \latex. 

Semoga template ini dapat membantu orang-orang yang ingin mencoba menggunakan 
\latex. Semoga template ini juga tidak berhenti disini dengan ada kontribusi 
dari para penggunanya. 
Kami juga ingin berterima kasih kepada Andreas Febrian, Lia Sadita, Fahrurrozi 
Rahman, Andre Tampubolon, dan Erik Dominikus atas kontribusinya dalam template 
ini. 

\vspace*{0.1cm}
\begin{flushright}
Depok, 27 Februari 2019\\[0.1cm]
\vspace*{1cm}
\penulis

\end{flushright}