%
% Halaman Abstrak
%
% @author  Andreas Febrian
% @version 1.00
%

\chapter*{Abstrak}

\vspace*{0.2cm}
{
	\setlength{\parindent}{0pt}
	
	\begin{tabular}{@{}l l p{10cm}}
		Nama&: & \penulis \\
		Program Studi&: & \program \\
		Judul&: & \judulIndo \\
		Pembimbing&: & \pembimbing \\
				  &  & \pembimbingDua \\
	\end{tabular}

	\bigskip
	\bigskip

	\emph{Benchmarking} adalah salah satu cara yang populer untuk melakukan eksperimen, terutama di bidang \emph{computational science}.
	\emph{Benchmarking} menyediakan kerangka kerja yang sistematis dan \emph{reproducible} untuk membandingkan performa antara implementasi.
	Alat \emph{benchmarking} menggunakan alat \emph{resource monitoring} untuk mengukur performa dari implementasi tertentu pada data tertentu.
	Di skripsi ini, implementasi alat \emph{resource monitoring} yang \emph{cross-platform} dan telah diuji di Linux dan macOS disediakan setelah melihat bahwa tidak ada alat yang berjalan di macOS.
	Alat-alat \emph{benchmarking} yang sudah ada ditinjau kemudian ditemukan bahwa tidak ada yang memenuhi semua kriteria yang ditentukan.
	Oleh karena itu, alat \emph{benchmarking} baru yang didesain untuk mendukung \emph{reproducibility} diajukan.
	Implementasi \emph{cross-platform} dari alat ini juga disediakan dan sudah diuji berjalan di Linux dan macOS.
	Evaluasi menunjukkan bahwa alat \emph{benchmarking} yang diajukan memenuhi hampir semua kriteria yang ditentukan kecuali pada poin adanya dokumentasi yang komprehensif dan pengujian.
	Alat yang diajukan menunjukkan potensi yang besar pada pengembangan selanjutnya untuk lebih menurunkan upaya untuk mencapai \emph{reproducibility}.

	\bigskip

	Kata kunci:\\
	\emph{Benchmarking}, \emph{resource monitoring}, komputasi terdistribusi, \emph{computational science}, \emph{reproducibility}
}

\newpage