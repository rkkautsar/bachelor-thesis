%
% Halaman Abstract
%
% @author  Andreas Febrian
% @version 1.00
%

\chapter*{Abstract}

\vspace*{0.2cm}
{
	\setlength{\parindent}{0pt}

	\begin{tabular}{@{}l l p{10cm}}
		Name&: & \penulis \\
		Study Program&: & \program \\
		Title&: & \judulInggris \\
		Supervisors&: & \pembimbing \\
				   &  & \pembimbingDua \\
	\end{tabular}

	\bigskip
	\bigskip

	Benchmarking is one popular way to conduct experiments, especially in the field of computational science.
	It provides a systematic, easily reproducible framework to compare performance between implementations.
	Benchmarking tool uses resource monitoring tool to measure the performance of selected implementations to selected test instances.
	We review the existing benchmarking tools and found none of them meet our requirements of an ideal benchmarking tool.
	We thus propose a new implementation of benchmarking tool designed for reproducible research.
	Our evaluation shows the proposed benchmarking tool came closer to meet our requirements, albeit not fully.
	Finally, the proposed tool also shows potential in future development to further reduce the effort to embrace reproducibility.

	\bigskip

	Key words:\\
	Benchmarking, distributed computing, computational science, reproducibility
}

\newpage