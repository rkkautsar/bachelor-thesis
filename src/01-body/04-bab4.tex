\chapter{\babEmpat}

\section{Overview}

To objectively compare two or more programs, an objective measurement is needed.
Resource consumption like CPU time elapsed, CPU instruction used, or peak memory usage is often considered as the go-to measurement for benchmarking in computational science.
The resources measured might include information regarding processes, threads, computation, memory, input/output, and files of a program \citep{juvePracticalResourceMonitoring2015}.

The measurement of these resource consumption is not only used in benchmarking.
Some of its usage includes but not limited to:
its daily usage in user program such as \textit{Task Manager} in Windows or \code{top} in Linux,
judging whether a program passes some threshold marks in education or competition field,
measuring the efficiency of a job scheduling system in High Throughput Computing (HTC) field [need citation],
selecting dataset for a competition,
and getting best-enough result from an iterative optimization algorithm after some desired time.
Because of this wide area of usage, there are many attempts to implement these measurement to achieve the best result.

\section{Monitoring Mechanism}

\citet{juvePracticalResourceMonitoring2015} distinguish resource monitoring into three different mechanism.
There are tradeoffs between these mechanisms and so it's often preferred to use a combination of more than one mechanism to achieve better results.

\subsection{Query}

The query approach works by querying resource usage information directly from the operating system.
To monitor the resource usage over time, this means the query is executed at some interval, effectively doing a polling mechanism.
More frequent polling will result in a more timely information but the overhead also increases.
Querying is often the easiest resource monitoring mechanism in term of implementation \citep{juvePracticalResourceMonitoring2015}.

It is the least intrusive mechanism, albeit the information received from this mechanism also immediately expires.
A resource usage surge could happen in between the queries, so this mechanism is not accurate.
In short, the query mechanism provides an easy to implement but inaccurate measurement of the resource monitored.

\code{getrusage()}\footnote{see \href{https://linux.die.net/man/2/getrusage}{\code{man getrusage}}} is a POSIX standard system call.
Unfortunately, this standard only specifies \code{ru\_utime} and \code{ru\_stime}, the user mode time spent and kernel mode time spent, respectively.
In practice, \code{getrusage()} includes more information, for example the \code{ru\_maxrss}, indicating the maximum resident set size.

A Linux specific feature that is often used for querying resource usage is the procfs\footnote{see \href{https://linux.die.net/man/5/proc}{\code{man 5 proc}}} pseudo-file system.
The information provided includes user time, system time, resident set size, number of threads, and many more.
On a side note, since the operating system always account these informations for all running processes, the clock resolution used is not too accurate.
The user time and system time in \code{/proc/[pid]/stat} is measured in system ticks, which turns out to be 10 milliseconds in a typical Linux system.
Recently, Linux control groups (CGroups)\footnote{see \href{http://man7.org/linux/man-pages/man7/cgroups.7.html}{\code{man 7 cgroups}}} is also used for querying similar or more comprehensive information for a group of processes.

A more universal query mechanism is using performance counters.
Most system nowadays---disregarding its operating system---has a hardware clock that has an incredibly accurate clock resolution, some even achieving nanoseconds resolution.
Querying this clock at the start and end of a program could measure what is called the wall clock time.
This measurement is less informative than CPU times measured by user time and system time, but still useful in spite of that.

The library \code{psutil}\footnote{\href{https://github.com/giampaolo/psutil}{https://github.com/giampaolo/psutil}} provides an abstraction for various resource queries that works in many operating systems.
This allows a cross-platform resource monitoring tool based on query mechanism to be developed.
As of now however, \first~can't find the relevant resource monitoring tool making use of this abstraction.

It is also need to be noted that this query mechanism cannot work with process tree reliably.
Because of the nature of polling, short living processes can be missed and not accounted to the final result.
This is a strong restriction because most of the computation in HTC or computational science in general often make use of parallelism or concurrency.


\subsection{Notification}

A more reliable mechanism to query a resource usage is to ask the system itself to report the usage on specific events.
This also produces less overhead compared to the query mechanism, although the information queried also immediately expires \citep{juvePracticalResourceMonitoring2015}.

\code{wait4()}\footnote{see \href{https://linux.die.net/man/2/wait4}{\code{man wait4}}} is a system call available in most UNIX that waits for a child process (and blocks the process calling this system call), then returns its \code{getrusage()} information when the process exits.
This one example of the notification mechanism in practice.

\code{inotify}\footnote{see \href{https://linux.die.net/man/7/inotify}{\code{man inotify}}} application programming interface (API) in Linux can also be leveraged.
This API allows one to listen for file system events, such as new file or directory created, existing files edited, a file is accessed, or a file is deleted.
Most operating system also have APIs similar to this, such as \code{fsevents} in OS X. Watchman\footnote{\href{https://facebook.github.io/watchman/}{https://facebook.github.io/watchman/}} is an open source tool by Facebook that abstracts these file system notification APIs.

Another useful notification is \code{forkstat}\footnote{\href{http://manpages.ubuntu.com/manpages/cosmic/en/man8/forkstat.8.html}{http://manpages.ubuntu.com/manpages/cosmic/en/man8/forkstat.8.html}}.
This tool notify system-wide \code{fork()}, \code{exec()}, and \code{exit()} system call activities.
Unfortunately, this tool needs superuser privilege because uses Linux netlink connector, a special socket for communication between kernel and user space.

A more powerful albeit intrusive notification can be achieved by using the UNIX \code{ptrace()}\footnote{see \href{https://linux.die.net/man/2/ptrace}{\code{man ptrace}}} system call.
\code{ptrace()}, often use for debugging, allows a process to intercept and modify system calls.
When a system call occurs, the kernel will check if the process is being traced, and if so, will notify the tracer.
With this system call one can observe, for example, the \code{fork()}, \code{exec()}, and \code{exit()} system call to track process trees.

Combining this notifications with the earlier query mechanism can result in a more powerful resource monitoring method.
For example, one can watch for filesystem events on a specific part of the filesystem, such as the \code{/tmp}, \code{/var}, or a specific work directory of the running application. Then when an event occurs, a procfs query is executed to catch the short-living process causing the event is executed.
This effectively allows the procfs query method to work more reliably compared to blindly polling the pseudo-file system.

\subsection{Interposition}

\citet{juvePracticalResourceMonitoring2015} defines this group of mechanism as the ones in which the monitor intercepts actions performed by the process.
This mechanism is highly intrusive because it actively changes the way the program runs.
\code{ptrace()} mentioned earlier also belongs to this group since the kernel stops the system call and forward an event to the monitor (in this case the tracer).
In general this method introduces huge overhead compared to the other mechanism.
In \code{ptrace()} for example, there need to be at least two context switches for every system call intercepted.

A less intrusive method compared to the one \code{ptrace()} used is function interposition.
In this mechanism, the monitor replace some original function by a wrapper that also record parameters and results of the original function \citep{juvePracticalResourceMonitoring2015}.
The environment variable \code{LD\_PRELOAD} in Linux and \code{DYLD\_INSERT\_LIBRARIES} in OS X provides an easy way to load arbitrary library before running a dynamically-linked program.
This way, one can write a wrapper function that will be called instead of the original function.

On the other hand, \code{ptrace()} is less practical because it interrupts each system call made by the program, but it is more robust and can be used in statically-linked programs.
\citet{kimPracticalEffectiveSandboxing2013} has shown that using the recent \code{seccomp/BPF}\footnote{see \href{http://man7.org/linux/man-pages/man2/seccomp.2.html}{\code{man seccomp}}} feature to filter out what system call is going to be intercepted results in an efficient system call interposition technique.

\code{seccomp} is a security feature in Linux (since 2.6.23) to allow only some set of system calls for a program.
Furthermore, Berkeley Packet Filter (\code{BPF}) is also supported since Linux 3.5 as a mean to configure the way \code{seccomp} filters the system call.
With the combination of \code{seccomp} and \code{BPF}, they can configure only some set of the system call to send a \code{ptrace()} event while the rest just continue normally, effectively reducing the overhead of system call interposition.
This feature---usually combined with Linux namespaces---can also be used to provide simple yet secure isolation for the program, as used by \textsc{nsjail}\footnote{see Section \ref{sec:resource.impl.nsjail}} and \textsc{firejail}\footnote{\href{https://github.com/netblue30/firejail/}{https://github.com/netblue30/firejail/}}.

With this interposition mechanism, one can reliably use the query mechanism.
For example, before the \code{exit()} system call actually happen, the \code{procfs} can be queried because the program itself can be stopped before the entry is removed from the pseudo-file system.
While this is the most accurate method compared to the others, this is also highly intrusive so caution must be exercised when using this method.

\section{Virtualization: More Reliable Measurements}

To strengthen the need to produce more reliable resource usage measurement, \citet{beyerReliableBenchmarkingRequirements2019} states isolation of runs as one of the requirements for an accurate and reliable benchmarking.
Specifically, they used Linux containers to provide a lightweight isolation.
Additionally, virtualization tools that are widely used also contributes to reproducibility by providing a programmatic way to pack program dependencies \citep{boettigerIntroductionDockerReproducible2015,kordonBenchKitToolMassive2014:2013}.

\subsection{Virtual Machines}

Virtual machines (VMs) use a technique called hypervisor-based virtualization.
Hypervisor is a software that creates and runs virtual machines \citep{scheepersVirtualizationContainerizationApplication2014}.
There are many of such hypervisor programs that are widely used, examples includes VirtualBox, Xen, Hyper-V, and many more.

Hypervisor manages the resource that will be used by the VMs.
When deployed, each VM run its own operating system with its own kernel.
All the dependencies needed by the program, is also included in the VM itself.
This results in a portable and isolated way to run a program.

Alas, this comprehensive virtualization comes with a cost.
The overhead of bootstrapping a VM is not negligible.
\citet{kordonBenchKitToolMassive2014} reports an average of 40 seconds of booting overhead when this virtualization is used in benchmarking.
\citet{scheepersVirtualizationContainerizationApplication2014} recommends using VMs (compared to containers) when it is important to distribute resources equally and performance should be less dependent from other running tasks.

\subsection{Containers}

A relatively new technique addressing the overhead issue is container-based virtualization.
Linux Containers (LXC)\footnote{\href{https://linuxcontainers.org/}{https://linuxcontainers.org/}} uses this technique to create an isolated environment without hosting multiple kernels like in hypervisor-based virtualization.
The implementation share the Linux kernel with the host machine.
This makes it more lightweight and particularly good for deploying many small isolated instances, as often the case in benchmarking.

Docker\footnote{\href{https://www.docker.com/}{https://www.docker.com/}} is an especially popular tool that utilizes LXC and offer an easy to use interface.
An application built with Docker extends upon existing Docker images with a reproducible \code{Dockerfile} configuration, stating each step needed to create the exact image.
With the key feature of sharing the Linux kernel, an OS image in Docker can achieve minimum size, with a notable example like Alpine Linux even reaching only 2-3 megabytes in size (its virtual machine images is ten times larger in size).

\citet{zhangComparativeStudyContainers2018} report a much less boot up latency---period from starting a VM or container to usable service---in Docker containers compared to VMs.
Booting a VM when there are about 250 other idle VMs takes more than 1000 seconds, while booting 512 Docker contains only takes 987 seconds (an average of 1.89 seconds per container).
This is also true for the amount of memory consumed in idle state.
A VM takes 0.23 GB compared to 0.03 GB of memory of Docker container in their experiment.

\section{Implementations}

We consider some existing implementation and discuss their method of measuring and limiting resource.

\subsection{\textsc{runsolver}}

\subsection{\textsc{runexec}}

\subsection{\textsc{kickstart}}

\subsection{\textsc{timeout}}

\subsection{\textsc{nsjail}}
\label{sec:resource.impl.nsjail}

\subsection{\textsc{isolate}}

\subsection{\textsc{psmon}}