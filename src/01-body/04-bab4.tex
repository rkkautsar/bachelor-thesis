\chapter{\babEmpat}

\section{Overview}
% - Measuring resource is often the entire point of benchmarking.
%   - Beyer et al. even designed almost the whole accurate and reliable benchmarking method around its measurement and limiting method (runexec).
% - Objective measurement gives birth to objective comparison.
% - And resource consumption is one obvious measurement for computational tasks.
% - These can include processes, threads, computation, memory, input/output, and files information. (Juve et al, 2015)
% - It is often also necessary to limit the resource consumption to some reasonable unit.
%   - Selecting data
%   - Job scheduling submission
%   - Competition
%   - Iterative optimization algorithm

To objectively compare two or more programs, an objective measurement is needed.
Resource consumption like CPU time elapsed, CPU instruction used, or peak memory usage is often considered as the go-to measurement for benchmarking in computational science.
The resources measured might include information regarding processes, threads, computation, memory, input/output, and files of a program \citep{juvePracticalResourceMonitoring2015}.

The measurement of these resource consumption is not only used in benchmarking.
Some of its usage includes but not limited to:
its daily usage in user program such as \textit{Task Manager} in Windows or \code{top} in Linux,
judging whether a program passes some threshold marks in education or competition field,
measuring the efficiency of a job scheduling system in High Throughput Computing (HTC) field [need citation],
selecting dataset for a competition,
and getting best-enough result from an iterative optimization algorithm after some desired time.
Because of this wide area of usage, there are many attempts to implement these measurement to achieve the best result.


\section{Approaches}

\citet{juvePracticalResourceMonitoring2015} distinguish resource monitoring to three different approaches:

\subsection{Query}
\subsection{Notification}
\subsection{Interposition}


\section{More reliable measurement}

\subsection{Virtualization}

\subsection{Containerization}


\section{Implementations}

We consider some existing implementation and discuss their method of measuring and limiting resource.

\subsection{\textsc{runsolver}}
\subsection{\textsc{runexec}}
\subsection{\textsc{kickstart}}
\subsection{\textsc{timeout}}
\subsection{\textsc{nsjail}}
\subsection{\textsc{isolate}}
\subsection{\textsc{psmon}}