\fancyhead[LE,RO]{\thepage}

\chapter{\chExperimentation}
\label{ch:experimentation}


First and foremost, \first~introduce the concepts and definitions that are used throughout the rest of this thesis.
This is to ensure alignment of the concepts of experiments, benchmarking, and reproducibility and how they relate with each other.
\First~discuss the practice of experimentation in science, specifically computational science, the concern of reproducibility, and how benchmarking as a systematic technique for experimenting may address this issue.
\Cref{sec:experimentation} considers how scientific method is used for experimentation in science and a few examples to motivate the needs of experimentation.
\Cref{sec:reproducibleBenchmarks} the reproducibility issue and benchmarking.
Finally, \Cref{sec:idealBenchmarkingTool} outlines the requirements \first~identified for an ideal benchmarking tool supporting reproducible research.


\section{Experimentation}
\label{sec:experimentation}

Science aims to further our understanding of why things happen in the natural world \citep{careyBeginnerGuideScientific2012}.
Investigating these questions can be achieved by following what is known as the scientific method.
This method is identical to the hypothetico-deductive method which is a combination of inductive and deductive reasoning \citep{wallimanResearchMethodsBasics2010a}.
\citeauthor{wallimanResearchMethodsBasics2010a} further defines the scientific method as follows:
\begin{enumerate}[noitemsep]
	\item identification or clarification of a problem;
	\item developing a hypothesis (testable theory) inductively from observations;
	\item charting their implications by deduction;
	\item practical or theoretical testing of the hypothesis;
	\item rejecting or refining it in the light of the results.
\end{enumerate}

\citet{careyBeginnerGuideScientific2012} provides a simpler definition for the scientific method as three simple steps:
 (1) observing, (2) explaining, and (3) testing.


Both definitions suggest that science advances by the means of trial and error.
The explaining-testing steps are repeated until the result is acceptable.
When a theory is falsified, another one is proposed and tested, until the most fitting theory is accepted \citep{wallimanResearchMethodsBasics2010a}.

Experimentation is one way to test these proposed theories or hypotheses.
Some would argue that experiments do not prove anything.
But on the other hand, experiments can falsify a theory and corroborate it, but not actually proving that the theory is true \citep{tichyShouldComputerScientists1998}.
This is similar to proof by contradiction commonly used in logic and mathematics.

Experimentation can also be used for induction to derive generalized conclusion from repeated observations.
Examples are the case of artificial neural network and SAT as already mentioned in \Cref{ch:introduction}.
Experiments show those hard-in-theory problems to be more tractable than expected and so the generalized conclusion that these problems are usable in real world applications is derived.

Not only in computer science, experimentation has been used in practice in many disciplines.
In particular, fields like physics and chemistry that involves observing and explaining physical objects often use experiments.
This is again shown by the examples we discuss in \Cref{ch:introduction}.

Besides for the objectives of validation and induction already mentioned above, there is a lot of reason to run an experiment.
\citet{montgomeryDesignAnalysisExperiments2013} mentioned some of the reasons as follows:
\begin {enumerate}[noitemsep]
	\item \textbf{Factor screening or characterization}. This is often the case when working with new systems or technologies to minimize wasted resource usage.
	\item \textbf{Optimization}. Finding the desirable settings of factors that result in desirable response.
	\item \textbf{Confirmation}. Same as the validation objective already mentioned.
	\item \textbf{Discovery}. Same as the induction objective already mentioned.
	\item \textbf{Robustness}. Finding the condition on which the response variables will degrade.
\end{enumerate}

\citet{montgomeryDesignAnalysisExperiments2013} also defines a guideline in designing an experiment:

\begin{enumerate}[noitemsep]
	\item Recognition of and statement of the problem
	\item Selection of the response variable
	\item Choice of factors, levels, and ranges
	\item Choice of experimental design
	\item Performing the experiment
	\item Statistical analysis of the data
	\item Conclusions and recommendations
\end{enumerate}

Steps 1-3 can be considered as the pre-experimental planning steps.
They also remark that Steps 2 and 3 are interchangeable in practice.
The experiment should also be done in an iterative way.
This is because generally \first~do not know the exact choices such as the factors, ranges, and measurements right away.
Instead, \first~learn it as \first~go along with the experiments.


\section{Reproducible Benchmarks}
\label{sec:reproducibleBenchmarks}

The term reproducibility is closely related with the term replicability.
\citet{drummondReplicabilityNotReproducibility2009} argues that replicability is the impoverished version of reproducibility which is not worth having.
While achieving it need a great effort, its purpose is only for preventing fraud.
The opposite is argued by \citet{vitekR3RepeatabilityReproducibility2012}.
Replicability is as important as reproducibility, since replicability provides baseline and facilitates extending and building on previous works.


\citet{goodmanWhatDoesResearch2016} suggests to define new terminologies for reproducibility to avoid the inconsistencies between replicability and reproducibility:
\begin{enumerate}
	\item \textbf{Methods reproducibility}. Ability to implement, as exactly as possible, the experimental and computational procedures, with the same data and tools, to obtain the same results.
	\item \textbf{Results reproducibility}. The production of corroborating results in a new study, having followed the same experimental methods.
	\item \textbf{Inferential reproducibility}. The making of knowledge claims of similar strength from a study replication or reanalysis.
\end{enumerate}


\citet{gundersenStateArtReproducibility2018} compares this definition to \citeauthor{drummondReplicabilityNotReproducibility2009} definition.
Methods reproducibility is the same as replicability, while the combination of both results and inferential reproducibility is reproducibility.
They also adopt the definition by \citet{goodmanWhatDoesResearch2016} specific to the field of Artificial Intelligence (AI).
While they define it specific in the AI field, it can also be applied to computational science in general with the modification in \Cref{def:reproducibility}.

\begin{mydef}[Reproducibility]
	\label{def:reproducibility}
	Degrees of reproducibility from highest to lowest, as generalized from \citet{gundersenStateArtReproducibility2018} is as follows:
	\begin{enumerate}
		\item \textbf{R1: Experiment Reproducible}. The results of an experiment are experiment reproducible when the execution of the same implementation produces the same results when executed on the same data.
		\item \textbf{R2: Data Reproducible}. The results of an experiment are data reproducible when an experiment is conducted that executes an alternative implementation that produces the same results when executed on the same data.
		\item \textbf{R3: Method Reproducible}. The results of an experiment are method reproducible when the execution of an alternative implementation produces the same results when executed on different data.
	\end{enumerate}
\end{mydef}


\textbf{R1} is the highest degree of reproducibility and \textbf{R3} the lowest, based on the documentation.
Achieving \textbf{R3} only requires documented method of the experiment.
\textbf{R2} extends \textbf{R3} to also require the documentation of the data used.
Finally, \textbf{R1} further extends \textbf{R2} by also requiring documented experiment implementation.
\Cref{def:reproducibility} will be used to state the degree of reproducibility throughout the rest of this \MakeLowercase{\type}.

\newcontent{
	For example, a machine learning research stating only that a neural network can solve a specific problem better than any other studied method is \textbf{R3} reproducible.
	If additionally the data set used to train and test the neural network is documented and can be used for other researches, it is also \textbf{R2} reproducible.
	Then if the detailed structure or even the source code of the neural network implementation is also documented so other researches can use it, the research is also \textbf{R1} reproducible.
}

\citet{gundersenStateArtReproducibility2018} surveyed 400 papers from 2013-2016 AI conferences and reported none of them achieve any degree of reproducibility in \Cref{def:reproducibility}.
They define a number of indicators to measure the degree of reproducibility for each factor: method, data, and implementation.
The degree of reproducibility of each paper as measured by the number of fulfilled indicators is only in the 24-26\% range.

\citet{collbergRepeatabilityComputerSystems2016} study further increase the consent of reproducibility.
The degree of reproduciblity of papers from ACM conferences and computer science journals in 2012-2013 is only up to 54\% even after contacting the authors and waiting for reply for up to two months.
They mention some of the reasons as licensing issues, no version tracking, the code is not ready for public, no backup, obsolete dependencies, and so on.

\newcontent{
	To achieve \textbf{R1}, the whole experiment process needs to be documented and designed to be reproducible.
	\citet{montgomeryDesignAnalysisExperiments2013} defined the steps needed to design an experiment:
	\begin{enumerate}
		\item \textbf{Planning}. As with designing an experiment, this includes selecting the measurements, the tool configurations, and benchmark instances.
		\item \textbf{Performing benchmarks \& collecting results}. This step can (and should) be automated to allow for repeated runs. The run itself is parallelizable so it can easily be run on multiple threads or machines.
		\item \textbf{Analysis}. The collected results are analyzed statistically and compared with each other.
		\item \textbf{Conclusion}. Based on the analysis step, there might be follow up runs with different configurations or measurements to correct the hypothesis.
	\end{enumerate}

	\begin{figure}
		\centering
		\ifdraft{
			\dummyfig{assets/diagrams/benchmarking.tikz}
		}{
			\tikzstyle {block} = [draw, text width=3.5cm, minimum height=1cm, align=center]
\tikzstyle {miniblock} = [draw=gray, dashed, text width=2.5cm, inner sep=1ex]

\resizebox{\textwidth}{!}{%
    \begin{tikzpicture}
        \node [block, inner sep=0pt] (planning) {
            \begin{tikzpicture}
                \matrix [row sep=1em] {
                    \node {Planning}; \\
                    \node [miniblock] {\footnotesize Implementation}; \\
                    \node [miniblock] {\footnotesize Dataset}; \\
                    \node [miniblock] {\footnotesize Limitation}; \\
                    \node [miniblock] {\footnotesize Metrics}; \\
                    \node {\dots}; \\
                };
            \end{tikzpicture}
        };
        \node[block, right=1cm of planning](benchmarking){Benchmarking};

        \node [block, right=1cm of benchmarking, inner sep=0pt] (analysis) {
            \begin{tikzpicture}
                \matrix [row sep=1em] {
                    \node {Analysis}; \\
                    \node [miniblock, outer sep=2ex] {\footnotesize Summaries}; \\
                    \node [miniblock, outer sep=2ex] {\footnotesize Ranks}; \\
                    \node [miniblock, outer sep=2ex] {\footnotesize Plots}; \\
                    \node {\dots}; \\
                };
            \end{tikzpicture}
        };

        \node[block, right=1cm of analysis](conclusion){Conclusion};

        \draw[-latex'] (planning) -- (benchmarking);
        \draw[-latex'] (benchmarking) -- (analysis);
        \draw[-latex'] (analysis) -- (conclusion);

    \end{tikzpicture}
}
		}
		\caption{Benchmarking as part of an experiment}
		\label{fig:benchmarking}
	\end{figure}

	As illustrated in \Cref{fig:benchmarking}, benchmarking is the execution step outputting result to be evaluated in the subsequent steps.
	\First~formally define the term \emph{benchmarking} in \Cref{def:benchmarking}.

	\begin{mydef}[Benchmarking]
		\label{def:benchmarking}
		The process of \emph{benchmarking} takes a set of \emph{tool configurations} $\bm{C}$ and a set of \emph{input instances} $\bm{I}$ that are subject to benchmarking.
		$\bm{R} \in \bm{C} \times \bm{I}$ is the set of \emph{benchmark runs} that is evaluated in this process.
		A tool configuration $\bm{C}_i$ is further defined as the product of a \emph{tool} and the space of its \emph{configuration}: $\bm{T_i} \times \bm{S_i}$.
	\end{mydef}

	Applying \Cref{def:benchmarking} to an example benchmarking of SAT solvers, a tool configuration $\bm{C}_1$ can be a parallel SAT solver and how many parallel process it use.
	Then the input instance $\bm{I}$ can be a set of benchmark in CNF (conjunctive normal form) such as the one used in SAT Competition 2018\footlink{http://sat2018.forsyte.tuwien.ac.at/benchmarks/}.
	The process of benchmarking in this context is the process executing the SAT solver with a specified configuration (e.g. two parallel process) with every instance from the specified benchmark from SAT Competition 2018.
	The computation time and memory usage is monitored and reported as the result of this benchmarking process.
	The result can then be used for analysis and inferring conclusions.

	This benchmarking process can be used for comparing different tools of the same domain, evaluating and comparing different features or configurations of a tool, or finding out the performance of a tool under different inputs \citep{beyerReliableBenchmarkingRequirements2019}.
}

By packing an experiment as a benchmark, the experiment can be repeated as much as needed because the process is automated.
It is a good way to simplify repeated experiments \citep{tichyShouldComputerScientists1998}.
When the benchmarking process itself is able to be shared, it has achieved the \textbf{R3} reproducibility, and can be called as a reproducible benchmark.
Then, depending on the effort to provide the data and software implementation, it can also achieve \textbf{R2} or even \textbf{R1} reproducibility.
This way, others can easily confirm or extends the benchmark with their own implementation and compare it with the original benchmarks.

Good benchmarking setup with documented hardware, software, configuration, and results is a good way to invite researchers to embrace reproducible research.
A tool for helping researchers setting up a reproducible benchmark is thus needed to reduce repeated work and lower the barrier of achieving reproducibility.


\section{An Ideal Benchmarking Tool}
\label{sec:idealBenchmarkingTool}

\First~define the following four requirements for the ideal benchmarking tool:
\begin{enumerate}[noitemsep]
	\item Extensible/configurable
	\item Minimal effort to setup
	\item Accurate and Reliable
	\item Reproducibility
\end{enumerate}

Requirements 1-4 above are compiled based on literature study and identifying strengths and weaknesses of existing solutions.
\First~consider six existing solutions described in \Cref{ch:existing}.
The list is also influenced by identifying what is important to have in a benchmarking tool based on a discussion with an active user of \textsc{benchmark-tool}, one particular tool we discuss in \Cref{ch:existing}.


\subsection{Extensible/Configurable}

Depending on the subject to be benchmarked, the benchmarking process can vary greatly in terms of the things measured and evaluated.
For example, the runtime measured for a sequential and parallel algorithm differs.
While CPU time is a more representative measurement of the actual time an algorithm use, wall-clock time captures the benefit of parallelism compared to the former.
Evaluation step also differs from one experiment to another.
One experiment might want to evaluate the performance of an algorithm, or the soundness of an algorithm, or even both.
This is why a benchmarking tool should be configurable to allow a wide range of usage across fields of research.

Another benefit from an extensible tool is community contribution.
For example, the popular python web framework Django\footlink{https://www.djangoproject.com/} has more than 35\,000 projects providing feature extensions for the framework to date\footlink{https://github.com/search?q=for+django}.
Additionally, an open sourced tool will also benefit from direct contribution to the project.
Taking another example from Django, today there are 1\,751 user from all over the world contributing directly to the project repository.


\subsection{Minimal Effort to Use}
\citet{vitekR3RepeatabilityReproducibility2012} mention that researches dealing with repeatability took considerable amount of effort to publish.
A benchmarking tool to help researchers with repeatability and reproducibility should not hinder the researchers with complex setup.
Setting up a benchmark and running it should be as easy as possible.

For example, requiring elevated privilege to the system for setting up the tools will often require additional complicated administration work, as such the case in TU Dresden cluster system.
Easy adoption of the tools can reduce the barrier to reproducible research.


\citet{nielsenTenUsabilityHeuristics2006} defines ten general usability heuristics in user interaction design.
This can be followed to ensure the usage is as easy as possible.
The ten principles are as follows:
\begin{enumerate}[noitemsep]
	\item Visibility of system status
	\item Match between system and the real world
	\item User control and freedom
	\item Consistency and standards
	\item Error prevention
	\item Recognition rather than recall
	\item Flexibility and efficiency of use
	\item Aesthetic and minimalist design
	\item Help users recognize, diagnose, and recover from errors
	\item Help and documentation
\end{enumerate}

Principle 10 from above in particular is very important in this context.
Help and documentation is needed to guide the user through setting up the benchmark.
A helpful guide can cut down the setup time and allow the user to instead focus on the important things such as choosing what to measure and how the evaluation should be executed.


\subsection{Accurate and Reliable}
To be useful, the benchmark itself needs to be accurate and reliable.
This is particularly challenging.
For example, since benchmark runs are often executed in parallel, they influence each other and thus the measurement can be inaccurate.
\citet{beyerReliableBenchmarkingRequirements2019} define several requirements of a reliable benchmark (and their solution) comprehensively.
The requirements are as follows:

\begin{enumerate}[noitemsep]
	\item Measure and limit resources accurately
	\item Terminate processes reliably
	\item Assign cores deliberately
	\item Respect nonuniform memory access
	\item Avoid swapping
	\item Isolate individual runs
\end{enumerate}

\subsection{Reproducibility}
The benchmarking tool should also embrace reproducibility.
This is to address the issue of reproducibility discussed earlier in \Cref{sec:reproducibleBenchmarks}.
The \textbf{R3} reproducibility can be achieved by making the benchmark setup sharable and can easily be re-run in another environment.
Additionally, to achieve \textbf{R2} or even \textbf{R1} reproducibility, the benchmarking tool should document the source of the data used and the version of the software implementation used in the benchmark, respectively.
