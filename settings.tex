%-----------------------------------------------------------------------------%
% Informasi Mengenai Dokumen
%-----------------------------------------------------------------------------%
%
% Judul laporan.
\var{\judul}{A Benchmarking Tool for Computational Reproducibility}
\Var{\Judul}{A Benchmarking Tool for Computational Reproducibility}
%
% Tulis kembali judul laporan, kali ini akan diubah menjadi huruf kapital
\var{\judulIndo}{Alat \emph{Benchmarking} untuk \emph{Reproducibility} Komputasional}
%
% Tulis kembali judul laporan namun dengan bahasa Ingris
\var{\judulInggris}{A Benchmarking Tool for Computational Reproducibility}

%
% Tipe laporan, dapat berisi Skripsi, Tugas Akhir, Thesis, atau Disertasi
\var{\type}{Thesis}
%
% Tulis kembali tipe laporan, kali ini akan diubah menjadi huruf kapital
\Var{\Type}{Thesis}

\var{\fullType}{Skripsi}
\Var{\FullType}{Skripsi}
%
% Tulis nama penulis
\var{\penulis}{Rakha Kanz Kautsar}
%
% Tulis kembali nama penulis, kali ini akan diubah menjadi huruf kapital
\Var{\Penulis}{Rakha Kanz Kautsar}
%
% Tulis NPM penulis
\var{\npm}{1506688784}
%
% Tuliskan Fakultas dimana penulis berada
\Var{\Fakultas}{Computer Science}
\var{\fakultas}{Computer Science}
%
% Tuliskan Program Studi yang diambil penulis
\Var{\Program}{Computer Science}
\var{\program}{Computer Science}
%
% Tuliskan tahun publikasi laporan
\Var{\bulan}{Juli}
\Var{\tahun}{2019}
%
% Tuliskan gelar yang akan diperoleh dengan menyerahkan laporan ini
\var{\gelar}{Sarjana Ilmu Komputer (Bachelor of Computer Science)}
%
% Tuliskan tanggal pengesahan laporan, waktu dimana laporan diserahkan ke
% penguji/sekretariat
\var{\tanggalPengesahan}{14 June 2019}
%
% Tuliskan tanggal keputusan sidang dikeluarkan dan penulis dinyatakan
% lulus/tidak lulus
\var{\tanggalLulus}{20 June 2019}
%
% Tuliskan pembimbing
\var{\pembimbing}{Lim Yohanes Stefanus, Ph.D}
\var{\pembimbingDua}{Dr. Johannes K. Fichte}
%
% Alias untuk memudahkan alur penulisan paa saat menulis laporan
\var{\saya}{Penulis}
\var{\First}{We}
\var{\first}{we}
\var{\firstposs}{our}
\var{\Firstposs}{Our}
\var{\OurBenchmarkingTool}{\textsc{ReproBench}}

%-----------------------------------------------------------------------------%
% Judul Setiap Bab
%-----------------------------------------------------------------------------%
%
% Berikut ada judul-judul setiap bab.
% Silahkan diubah sesuai dengan kebutuhan.
%
\Var{\kataPengantar}{Preface}
\Var{\chIntroduction}{Introduction}
\Var{\chExperimentation}{Experimentation and Benchmarking}
\Var{\chResource}{Measuring and Limiting Resource}
\Var{\chExisting}{Benchmarking Tools}
\Var{\chImplementation}{\OurBenchmarkingTool: A New Benchmarking Tool}
\Var{\chEvaluation}{Evaluation}
\Var{\chConclusion}{Conclusion and Future Work}


\usetikzlibrary{shapes,arrows,positioning,fit,backgrounds,matrix,er}

\newcommand{\footlink}[1]{\footnote{See \url{#1}}}

\definecolor{newcontentcolor}{rgb}{0.2,0.5,1}
\definecolor{newstylecolor}{rgb}{0.1,0.3,0.6}
% \newcommand{\newcontent}[1]{{\leavevmode\color{newcontentcolor} #1}}
% \newcommand{\newstyle}[1]{{\leavevmode\color{newstylecolor} #1}}
\newcommand{\newcontent}[1]{{#1}}
\newcommand{\newstyle}[1]{{#1}}

\newtheorem{mydef}{Definition}
